Conformal prediction, first introduced by Vovk et al. \cite{vovk2005algorithmic}, provides a powerful framework for uncertainty quantification. Unlike traditional methods that yield point estimates, conformal prediction produces prediction intervals or sets with guaranteed marginal coverage, regardless of the underlying data distribution or the model used. The only requirement is the assumption of exchangeability of the data. The original formulation, known as "full" conformal prediction, was computationally intensive. The development of "split" conformal prediction \cite{papadopoulos2002inductive} made the method more practical by splitting the data into a training set and a calibration set, significantly reducing the computational cost.

The primary limitation of standard conformal prediction is the assumption of exchangeability. This assumption is often violated in real-world applications, especially those involving time series data, where the data distribution can change over time. To address this, the concept of weighted conformal prediction was introduced \cite{tibshirani2019conformal}. By assigning weights to the calibration samples, this method can account for non-exchangeability. For instance, in a time series setting, more recent data points can be given higher weights to reflect their greater relevance to future predictions. The theoretical guarantees for weighted conformal prediction show that the coverage gap can be bounded, and this bound is related to the total variation distance between the distributions of the calibration and test data.

The application of conformal prediction to time series data has been an active area of research \cite{stankeviciute2021conformal, gibbs2021adaptive, zaffran2022adaptive}. Besides weighted conformal prediction, other techniques have been proposed to handle non-exchangeable data. For example, some methods use a rolling window approach to update the calibration set over time, while others use more sophisticated models to estimate the non-conformity scores \cite{xu2021conformal}.

Panel data, which combines a time-series dimension with a cross-sectional dimension, presents unique challenges. While there has been some work on applying conformal prediction to panel data \cite{dunn2022distribution, chernozhukov2021distributional, oliveira2022split}, it is not as extensive as the work on standard time series data. Our work, QUPEC, builds upon the weighted conformal prediction framework and extends it to the panel data setting by incorporating both temporal and cross-sectional information into the weighting scheme. This allows the method to adapt to the specific characteristics of panel data, where dependencies exist both over time and across entities.
