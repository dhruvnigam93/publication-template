The results of our experiments highlight a key trade-off in uncertainty quantification: the balance between coverage and the width of prediction intervals. QUPEC successfully reduces the coverage gap, providing more reliable uncertainty estimates. However, this comes at the cost of wider intervals. This is a desirable outcome in many real-world applications where the cost of an uncovered prediction is high. For example, in medical diagnosis or financial risk assessment, it is often preferable to have a wider interval that contains the true value with high probability than a narrow interval that is more likely to be wrong.

The improved coverage of QUPEC can be attributed to its ability to effectively leverage both temporal and cross-sectional information. By assigning higher weights to more recent and more similar entities, QUPEC can better adapt to the non-exchangeable nature of panel data. The standard conformal prediction method, which treats all data points as equally important, is not well-suited for this type of data, as evidenced by its larger coverage gap. The time-weighted method is an improvement, but it still fails to capture the cross-sectional dependencies that are present in panel data.

The choice of the hyperparameters $\beta_{time}$ and $\beta_{entity}$ is crucial for the performance of QUPEC. In our experiments, we used cross-validation to select these parameters.
