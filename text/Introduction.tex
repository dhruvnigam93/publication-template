
In the last decade, supervised machine learning has have gotten increasingly better at making predictions due to better algorithms, reduction in the cost of computation and an increase in the amount of data available to train these models. Consequently, it's use in industries like finance, e-commerce, logistics and healthcare has increased. While significant progress continues to be made in this area, quantifying uncertainty in these predictions remains a challenging problem \cite{abdar2021review, malinin2019uncertainty}. In many high-risk decision-making tasks like medical diagnosis and financial risk management, the ability to quantify the uncertainty in the predictions is as important as their accuracy. Statistical learning models like Gaussian process regressions \cite{schulz2018tutorial} are able to produce statistically robust uncertainty estimates if the assumptions made by the underlying models about the data generative process are satisfied. However, they are seldom as accurate \cite{breiman2001statistical}.

Conformal prediction \cite{vovk2022algorithmic} is a promising technique for quantifying uncertainty in predictions made by any supervised machine learning model. It does so by producing prediction intervals with guaranteed coverage. However, challenges arise when applying it to panel data, where the assumption of exchangeability is often violated due to temporal dynamics and cross-sectional heterogeneity. While extensions of conformal prediction to temporal data which exhibits non-exchangability have been proposed \cite{barber2022conformall}, they do not account for the cross-sectional heterogeneity present in panel data.

In this paper we propose a framework for quantifying uncertainty in predictions in panel data using weighted conformal prediction. We use the temporal dynamics and cross-sectional heterogeneity inherent to panel data to assign weights to the calibration points in the weighted conformal prediction framework to account for the non-exchangeability of the data by using a kernel-based weighting scheme that combines temporal and cross-sectional distances between the calibration points and the test point. Through experiments on real-world e-commerce data, we show that this approach provides more accurate and reliable prediction intervals then using the standard conformal prediction method or weighted conformal prediction that only accounts for the temporal structure of the data.

